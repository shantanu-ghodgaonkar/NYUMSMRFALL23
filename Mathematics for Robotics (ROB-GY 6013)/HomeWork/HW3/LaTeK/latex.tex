\documentclass[a4paper]{article}
\usepackage{enumitem, amsmath, gensymb, graphicx, caption, amssymb, geometry, fancyhdr, arydshln, adjustbox}

\geometry{left=1in, right=1in, top=1in, bottom=1in}
\pagestyle{fancy}

\newcommand{\myName}{\textbf{Shantanu Ghodgaonkar}\\\textit{Univ ID}: N11344563\\\textit{Net ID}: sng8399\\\textit{Ph.No.}: +1 (929) 922-0614}
\newlist{qalist}{description}{1}
\setlist[qalist]{style=unboxed,leftmargin=0.5cm,labelwidth=2.5cm}


\title{Homework 2 Answers : ROB-GY 6003}
\author{\myName}
\date{\today}

\fancyhead{} % Clear existing header settings 
\fancyhead[L]{\today}
\fancyhead[R]{N11344563}


\begin{document}
	
%	\begin{titlepage}
%	    \centering
%	    \vspace{2cm}
%	    \Huge\textbf{Mathematics for Robotics \\ ROB-GY 6103 \\ Homework 3 Answers}
%	    \vspace{1cm}
%	    \\ \Large \today
%	    \vfill
%	    \Large \myName
%	\end{titlepage}
	
	\begin{qalist}			
%		\item[Question: 1.(a)] \setcounter{equation}{0} Nagy, Page 136, Prob. 4.4.3
%		\item[Answer:] Given, $V = {\mathbb{P}}_{2}$ with the ordered basis $\mathcal{S} = \left( {p}_{0} = 1,\;{p}_{1} = x,\;{p}_{2} = {x}^{2}\right)$ 
%		
%		And the given polynomial is $r(x) = 2 + 3x - {x}^{2}$.
%		
%		
%		$\therefore$ the components of $r(x)$ in basis $\mathcal{S}$ is 
%		
%		\begin{equation}
%			{\text{r}}_{\mathcal{S}} = \begin{bmatrix} 2 \\ 3 \\ -1 \end{bmatrix}
%		\end{equation}
%		
%		\item[Question: 1.(b)] \setcounter{equation}{0} %Nagy, Page 136, Prob. 4.4.3
%		\item[Answer:] Given, $V = {\mathbb{P}}_{2}$ with the ordered basis $\mathcal{Q} = \left( {q}_{0} = 1,\;{q}_{1} = 1 - x,\;{q}_{2} = x + {x}^{2}\right)$. 
%		
%		From the definition of the basis $\mathcal{Q}$, we can rewrite it in terms of basis $\mathcal{S}$ as, 
%		
%		${q}_{0_{\mathcal{S}}} = \begin{bmatrix}1 \\ 0 \\ 0\end{bmatrix}$, ${q}_{1_{\mathcal{S}}} = \begin{bmatrix}1 \\ -1 \\ 0\end{bmatrix}$ and ${q}_{2_{\mathcal{S}}} = \begin{bmatrix}0 \\ 1 \\ 1\end{bmatrix}$ $\Rightarrow {Q}_{\mathcal{S}} = \begin{bmatrix}1 & 1 & 0 \\ 0 & -1 & 1 \\ 0 & 0 & 1\end{bmatrix}$
%		
%		And the given polynomial is $r(x) = 2 + 3x - {x}^{2}$.
%		\begin{equation}
%			\Rightarrow {\text{r}}_{\mathcal{S}} = \begin{bmatrix} 2 \\ 3 \\ -1 \end{bmatrix}
%		\end{equation}
%		
%		Now, we need to find numbers ${\tilde{r}}_{0}$, ${\tilde{r}}_{1}$ and ${\tilde{r}}_{2}$ such that,
%		\begin{equation}{\text{r}}_{\mathcal{S}} = {\tilde{r}}_{0}{\text{q}}_{{0}_{\mathcal{S}}} + {\tilde{r}}_{1}{\text{q}}_{{1}_{\mathcal{S}}} + {\tilde{r}}_{2}{\text{q}}_{{2}_{\mathcal{S}}}\end{equation}
%		
%		\begin{equation}
%			\Rightarrow \begin{bmatrix} 2 \\ 3 \\ -1 \end{bmatrix} = 
%			{\tilde{r}}_{0}\begin{bmatrix}1 \\ 0 \\ 0\end{bmatrix} + 
%			{\tilde{r}}_{1}\begin{bmatrix}1 \\ -1 \\ 0\end{bmatrix} + 
%			{\tilde{r}}_{2}\begin{bmatrix}0 \\ 1 \\ 1\end{bmatrix}
%		\end{equation}
%		
%		${Eq}^{n}$ can be rewritten as
%		
%		\begin{equation}
%			\begin{bmatrix}1 & 1 & 0 \\ 0 & -1 & 1 \\ 0 & 0 & 1\end{bmatrix} \begin{bmatrix}{\tilde{r}}_{0} \\ {\tilde{r}}_{1} \\ {\tilde{r}}_{2}\end{bmatrix} = \begin{bmatrix} 2 \\ 3 \\ -1 \end{bmatrix}
%		\end{equation}
%		
%		Applying the Gauss method,
%		\begin{equation}
%			\left[\begin{array}{ccc|c}1 & 1 & 0 & 2\\ 0 & -1 & 1 & 3\\ 0 & 0 & 1 & -1\end{array}\right]
%		\end{equation}
%		
%		${R}_{2} \rightarrow (-1){R}_{2}$
%		\begin{equation}
%			\left[\begin{array}{ccc|c}1 & 1 & 0 & 2\\ 0 & 1 & -1 & -3\\ 0 & 0 & 1 & -1\end{array}\right]
%		\end{equation}
%		
%		${R}_{1} \rightarrow {R}_{1} + (-1){R}_{2}$
%		\begin{equation}
%			\left[\begin{array}{ccc|c}1 & 0 & 1 & 5\\ 0 & 1 & -1 & -3\\ 0 & 0 & 1 & -1\end{array}\right]
%		\end{equation}
%		
%		${R}_{2} \rightarrow {R}_{2} + {R}_{3}$
%		\begin{equation}
%			\left[\begin{array}{ccc|c}1 & 0 & 1 & 5\\ 0 & 1 & 0 & -4\\ 0 & 0 & 1 & -1\end{array}\right]
%		\end{equation}
%
%		${R}_{1} \rightarrow {R}_{1} + (-1){R}_{3}$
%		\begin{equation}
%			\left[\begin{array}{ccc|c}1 & 0 & 0 & 6\\ 0 & 1 & 0 & -4\\ 0 & 0 & 1 & -1\end{array}\right]
%		\end{equation}		
%		
%		Hence, the solution is, 
%		\begin{equation}
%			{\text{r}}_{q} = \begin{bmatrix}6 \\ -4 \\ -1\end{bmatrix}
%		\end{equation}		
%
%		\item[Question: 2.] \setcounter{equation}{0}
%		\item[Answer:] Given the matrix, 
%			\begin{equation}
%				{A}_{3} = \begin{bmatrix} 1 & 4 & 10 \\ 0 & 2 & 0 \\ 0 & 0 & 3\end{bmatrix}
%			\end{equation}
%			By definition, we know that, $\exists v \neq 0 \text{ s.t. } Av = \lambda v \; \Rightarrow (\lambda I - A) v = 0 \; \Leftrightarrow det(\lambda I - A) = 0$ 
%			\begin{equation}
%				\Rightarrow \lambda I - A = \begin{bmatrix}\lambda-1 & -4 & -10 \\ 0 & \lambda - 2 & 0 \\ 0 & 0 & \lambda -3\end{bmatrix}
%			\end{equation}
%			
%			\begin{equation}
%				\Rightarrow det(\lambda I - A) = (\lambda - 1)\cdot(\lambda - 2)\cdot(\lambda - 3) = 0
%			\end{equation}
%						
%			\begin{equation}
%				\therefore {\lambda}_{1} = 1, {\lambda}_{2} = 2 , {\lambda}_{3} = 3
%			\end{equation}
%			
%			Now, we shall apply the known relation $A{v}^{i} = {\lambda}_{i}{v}^{i}$ $\Rightarrow (A - {\lambda}_{i}I){v}^{i} = 0$ 
%			\begin{equation} \label{eq:q2eigVecEqn}
%				\begin{bmatrix}1 - {\lambda}_{i} & 4 & 10 \\ 0 & 2 - {\lambda}_{i} & 0 \\ 0 & 0 & 3 - {\lambda}_{i}\end{bmatrix} 
%				\begin{bmatrix}{v}^{i}_{1} \\ {v}^{i}_{2} \\ {v}^{i}_{3}\end{bmatrix} = 
%				\begin{bmatrix}0 \\ 0 \\ 0\end{bmatrix}
%			\end{equation}
%			
%			From ${Eq}^{n}~\ref{eq:q2eigVecEqn}$, substituting ${\lambda}_{1} = 1$
%			\begin{equation}
%				\begin{bmatrix}0 & 4 & 10 \\ 0 & 1 & 0 \\ 0 & 0 & 2\end{bmatrix} 
%				\begin{bmatrix}{v}^{1}_{1} \\ {v}^{1}_{2} \\ {v}^{1}_{3}\end{bmatrix} = 
%				\begin{bmatrix}0 \\ 0 \\ 0\end{bmatrix}
%				\Rightarrow
%				\begin{array}{c}
%					4{v}^{1}_{2} + 10{v}^{1}_{3} = 0 \\
%					{v}^{1}_{2} = 0 \\
%					2{v}^{1}_{3} = 0
%				\end{array}
%				\Rightarrow \begin{array}{c}{v}^{1}_{1} = \text{any value} \\ {v}^{1}_{2} = 0 \\ {v}^{1}_{3} = 0\end{array} \Rightarrow {v}^{1}= \begin{bmatrix}1 \\ 0 \\ 0\end{bmatrix}
%			\end{equation}
%			
%			From ${Eq}^{n}~\ref{eq:q2eigVecEqn}$, substituting ${\lambda}_{2} = 2$
%			\begin{equation}
%				\begin{bmatrix}-1 & 4 & 10 \\ 0 & 0 & 0 \\ 0 & 0 & 1\end{bmatrix} 
%				\begin{bmatrix}{v}^{2}_{1} \\ {v}^{2}_{2} \\ {v}^{2}_{3}\end{bmatrix} = 
%				\begin{bmatrix}0 \\ 0 \\ 0\end{bmatrix}
%				\Rightarrow
%				\begin{array}{c}
%					-{v}^{2}_{1} + 4{v}^{2}_{2} + 10{v}^{2}_{3} = 0 \\
%					0 = 0 \\
%					{v}^{2}_{3} = 0
%				\end{array}
%				\Rightarrow \begin{array}{c}{v}^{2}_{1} = 4{v}^{2}_{2} \\ {v}^{2}_{2}=\text{any value}\\ {v}^{2}_{3}=0\end{array} \Rightarrow {v}^{2} = \begin{bmatrix}4 \\ 1 \\ 0\end{bmatrix}
%			\end{equation}
%			
%			From ${Eq}^{n}~\ref{eq:q2eigVecEqn}$, substituting ${\lambda}_{3} = 3$
%			\begin{equation}
%				\begin{bmatrix}-2 & 4 & 10 \\ 0 & -1 & 0 \\ 0 & 0 & 0\end{bmatrix} 
%				\begin{bmatrix}{v}^{3}_{1} \\ {v}^{3}_{2} \\ {v}^{3}_{3}\end{bmatrix} = 
%				\begin{bmatrix}0 \\ 0 \\ 0\end{bmatrix}
%				\Rightarrow
%				\begin{array}{c}
%					-2{v}^{3}_{1} + 4{v}^{3}_{2} + 10{v}^{3}_{3} = 0 \\
%					-{v}^{3}_{2} = 0 \\
%					0 = 0
%				\end{array}
%				\Rightarrow \begin{array}{c}{v}^{3}_{1} = 5{v}^{3}_{3} \\ {v}^{3}_{2}=0\\ {v}^{3}_{3}=\text{any value}\end{array} \Rightarrow {v}^{3} = \begin{bmatrix}5 \\ 0 \\ 1\end{bmatrix}
%			\end{equation}
%			
%			Now we shall verify that ${v}^{1}$, ${v}^{2}$ and ${v}^{3}$ are \textit{Linearly Independant},
%			\begin{equation}
%				{\alpha}_{1}{v}^{1}+{\alpha}_{2}{v}^{2}+{\alpha}_{3}{v}^{3} = 0
%			\end{equation}
%			
%			\begin{equation}
%				{\alpha}_{1}\begin{bmatrix}1 \\ 0 \\ 0\end{bmatrix}+
%				{\alpha}_{2}\begin{bmatrix}4 \\ 1 \\ 0\end{bmatrix}+
%				{\alpha}_{3}\begin{bmatrix}5 \\ 0 \\ 1\end{bmatrix}
%				= 0
%			\end{equation}
%			
%			\begin{align}
%				{\alpha}_{1} + 4{\alpha}_{2} + 5{\alpha}_{3} = 0 \\
%				{\alpha}_{2} = 0 \\
%				{\alpha}_{3} = 0
%			\end{align}
%			
%			Solving above system of equations gives us ${\alpha}_{1} = 0$, ${\alpha}_{2} = 0$ and ${\alpha}_{3} = 0$. \\
%			$\therefore$ ${v}^{1}$, ${v}^{2}$ and ${v}^{3}$ are \textit{Linearly Independant}.
		
%			
%		\item[Question: 3.] \setcounter{equation}{0}
%		\item[Answer:] Given the matrix, 
%			\begin{equation}
%				{A}_{4} = \begin{bmatrix} 3 & 1 & 0 \\ 0 & 3 & 0 \\ 0 & 0 & 2\end{bmatrix}
%			\end{equation}
%			By definition, we know that, $\exists v \neq 0 \text{ s.t. } Av = \lambda v \; \Rightarrow (\lambda I - A) v = 0 \; \Leftrightarrow det(\lambda I - A) = 0$ 
%			\begin{equation}
%				\Rightarrow \lambda I - A = \begin{bmatrix}\lambda-3 & -1 & 0 \\ 0 & \lambda - 3 & 0 \\ 0 & 0 & \lambda -2\end{bmatrix}
%			\end{equation}
%			
%			\begin{equation}
%				\Rightarrow det(\lambda I - A) = (\lambda - 3)\cdot(\lambda - 3)\cdot(\lambda - 2) = 0
%			\end{equation}
%			
%			\begin{equation}
%				\therefore {\lambda}_{1} = 3, {\lambda}_{2} = 2
%			\end{equation}			
%			
%			Now, we shall apply the known relation $A{v}^{i} = {\lambda}_{i}{v}^{i}$ $\Rightarrow (A - {\lambda}_{i}I){v}^{i} = 0$ 
%			\begin{equation}\label{eq:q3eigVecEqn}
%				\begin{bmatrix}3 - {\lambda}_{i} & 1 & 0 \\ 0 & 3 - {\lambda}_{i} & 0 \\ 0 & 0 & 2 - {\lambda}_{i}\end{bmatrix} 
%				\begin{bmatrix}{v}^{i}_{1} \\ {v}^{i}_{2} \\ {v}^{i}_{3}\end{bmatrix} = 
%				\begin{bmatrix}0 \\ 0 \\ 0\end{bmatrix}
%			\end{equation}
%			
%			From ${Eq}^{n}~\ref{eq:q3eigVecEqn}$, substituting ${\lambda}_{1} = 3$
%			\begin{equation}
%				\begin{bmatrix}0 & 1 & 0 \\ 0 & 0 & 0 \\ 0 & 0 & -1\end{bmatrix} 
%				\begin{bmatrix}{v}^{1}_{1} \\ {v}^{1}_{2} \\ {v}^{1}_{3}\end{bmatrix} = 
%				\begin{bmatrix}0 \\ 0 \\ 0\end{bmatrix}
%				\Rightarrow
%				\begin{array}{c}
%					{v}^{1}_{2} = 0 \\
%					0 = 0 \\
%					-{v}^{1}_{3} = 0
%				\end{array}
%				\Rightarrow \begin{array}{c}{v}^{1}_{1} = \text{any value} \\ {v}^{1}_{2} = 0 \\ {v}^{1}_{3} = 0\end{array} \Rightarrow {v}^{1}= \begin{bmatrix}1 \\ 0 \\ 0\end{bmatrix}
%			\end{equation}
%			
%			From ${Eq}^{n}~\ref{eq:q3eigVecEqn}$, substituting ${\lambda}_{2} = 2$
%			\begin{equation}
%				\begin{bmatrix}1 & 1 & 0 \\ 0 & 1 & 0 \\ 0 & 0 & 0\end{bmatrix} 
%				\begin{bmatrix}{v}^{2}_{1} \\ {v}^{2}_{2} \\ {v}^{2}_{3}\end{bmatrix} = 
%				\begin{bmatrix}0 \\ 0 \\ 0\end{bmatrix}
%				\Rightarrow
%				\begin{array}{c}
%					{v}^{2}_{1} = 0 \\
%					{v}^{2}_{2}= 0 \\
%					0 = 0
%				\end{array}
%				\Rightarrow \begin{array}{c}{v}^{2}_{1} = 0 \\ {v}^{2}_{2} = 0 \\ {v}^{2}_{3} = \text{any value}\end{array} \Rightarrow {v}^{2}= \begin{bmatrix}0 \\ 0 \\ 1\end{bmatrix}
%			\end{equation}
%			
%			Consider the set of eigenvectors $\mathcal{V} = \{{v}^{1}, {v}^{2}\}$. Firstly,
%			\begin{equation} \label{q3EVLIStmnt}
%				\text{We know that any set of eigenvectors of a given matrix are Linearly Independent. So, ${v}^{1}, {v}^{2}$ are \textit{L.I.}}
%			\end{equation}
%			
%			Secondly, consider a linear combination $x$ such that, 
%			\begin{equation}\label{q3SpanExp}
%				\left\{x\in\mathbb{R}\;|\;\exists \; {\alpha}_{1}, {\alpha}_{2} \in \mathbb{R}, {v}^{1}, {v}^{2} \in \mathcal{V} \;;\;s.t.\;x = {\alpha}_{1}{v}^{1} + {\alpha}_{2}{v}^{2}\right\}
%			\end{equation}
%			By observing above ${Eq}^{n}$~\ref{q3SpanExp} we can see that it holds true for all cases.
%			
%			$\therefore$ based on the ${Eq}^{n}$ ~\ref{q3EVLIStmnt} and ~\ref{q3SpanExp} we can say that $\mathcal{V}$ forms a basis for ${\mathbb{R}}^{3}$.
%
%		\item[Question: 4.] \setcounter{equation}{0} 
%		\item[Answer:] We are given two similar square matrices $A$ and $B$ such that,
%			\begin{equation}\label{q4similarityRelation}B = {P}^{-1} A P\end{equation}
%			
%			Consider the characteristic relation of matrix $B$, 
%			\begin{equation}\label{q4BcharReln1} det(\lambda I - B) \end{equation}
%			
%			Apply ${Eq}^{n}~\ref{q4similarityRelation}$ in ${Eq}^{n}~\ref{q4BcharReln}$, 
%			\begin{equation}\label{q4BcharReln2}
%				\Rightarrow  det(\lambda I - B) =  det(\lambda I - {P}^{-1} A P) 
%			\end{equation}
%			
%			The identity matrix $I$ can also be written as ${P}^{-1} I P$. Substituting in ${Eq}^{n}$ ~\ref{q4BcharReln2}, 
%			\begin{equation}\label{q4BcharReln3}
%				\Rightarrow  det(\lambda I - B) =  det(\lambda {P}^{-1} I P - {P}^{-1} A P) 
%			\end{equation}
%			
%			Now we shall take ${P}^{-1}$ and $P$ out common, 
%			
%			\begin{equation}\label{q4BcharReln4}
%				\Rightarrow  det(\lambda I - B) =  det({P}^{-1} (\lambda I - A) P) 
%			\end{equation}
%			
%			We know that for compatible square matrices $A$ and $B$, $det(AB) = det(A)det(B)$. So ${Eq}^{n}$ ~\ref{q4BcharReln4} becomes, 
%			\begin{equation}\label{q4BcharReln5}
%				\Rightarrow  det(\lambda I - B) =  det({P}^{-1}) det(\lambda I - A) det(P) 
%			\end{equation}
%			
%			Cancelling out $ det({P}^{-1})$ by $det(P)$
%			\begin{equation}\label{q4BcharReln6}
%				\Rightarrow  det(\lambda I - B) =  det(\lambda I - A)
%			\end{equation}
%			
%			When the relation in ${Eq}^{n}$ ~\ref{q4BcharReln6} is equated to zero, we prove that the two matrices have the same eigenvalues as well as the characteristic equations. I.E.,
%			\begin{equation}\label{q4BcharReln6}
%				\Rightarrow  det(\lambda I - B) =  det(\lambda I - A) = 0
%			\end{equation}
%			
%			\textbf{Q.E.D.}
%
%		\item[Question: 5.] \setcounter{equation}{0} \textbf{$P$ is the eigenvector matrix}
%		\item[Answer:] Given a matrix ${A}_{3}$ such that,
%		
%			\begin{equation} \label{q5Given}
%				{A}_{3} = \begin{bmatrix}1 & 4 & 10 \\ 0 & 2 & 0 \\ 0 & 0 & 3\end{bmatrix}
%			\end{equation}
%			
%			From Question: 2. We found the eigenvalues to be ${\lambda}_{1} = 1$, ${\lambda}_{2} = 2$, ${\lambda}_{3} = 3$. 
%			
%			Using these eigenvalues, we can create the diagonal martix, $\Lambda = diag({\lambda}_{1}, {\lambda}_{2}, {\lambda}_{3})$,
%			\begin{equation}\label{q5DiagMat}
%				\Lambda = \begin{bmatrix}1 & 0 & 0 \\ 0 & 2 & 0 \\ 0 & 0 & 3\end{bmatrix}
%			\end{equation}
%			
%%			To show similarity between $A$ and $\Lambda$, we need to find a matrix $P$ such that $\Lambda = P A {P}^{-1}$
%			
%			Consider a matrix $P = \begin{bmatrix}\end{bmatrix}$ such 
%			
%		\item[Question: 6.(a)] \setcounter{equation}{0} 
%		\item[Answer:] Given, a vector space $(\mathcal{X}, \mathbb{R})$ where $\mathcal{X}$ is a set of 2 x 2 matrices with real coefficients.
%		
%		An operation is defined $L : \mathcal{X} \rightarrow \mathcal{X}$ by 
%		\begin{equation}\label{q6aOptDefn}
%			L(M) = \frac{1}{2} (M + {M}^{T})
%		\end{equation}
%		Where $M \in \mathcal{X}$ is a 2 x 2 real matrix
%		
%		The operator $L$ will be considered a \textit{Linear Operator} if, 
%		\begin{equation}\label{q6aLinOpStmt}
%			\forall x,y \in \mathcal{X} \;,\; \alpha, \beta \in \mathbb{R}\;|\;L(\alpha x + \beta y) = \alpha L(x) + \beta L(y)
%		\end{equation}
%		Applying above Statement ~\ref{q6aLinOpStmt} to ${Eq}^{n}$ ~\ref{q6aOptDefn},
%		\begin{equation}\label{q6aPrfStart}
%			L(\alpha x + \beta y) = \frac{1}{2}((\alpha x + \beta y) + {(\alpha x + \beta y)}^{T})
%		\end{equation}
%		
%		Applying the property of sum of transpose of two matrices \\ \textit{(i.e. for two matrices $A$ and $B \rightarrow$ ${(A+B)}^{T} = {A}^{T} + {B}^{T}$)}
%		
%		\begin{equation}\label{q6aPrfTranspose}
%			L(\alpha x + \beta y) = \frac{1}{2}(\alpha x + \beta y + {\alpha x}^{T} + {\beta y}^{T})
%		\end{equation}
%		
%		\begin{equation}\label{q6aPrfRearrange}
%			%L(\alpha x + \beta y) 
%			= \frac{1}{2}(\alpha x + {\alpha x}^{T} + \beta y  + {\beta y}^{T})
%		\end{equation}
%		
%		\begin{equation}\label{q6aPrfTakingCommon}
%			%L(\alpha x + \beta y) 
%			= \alpha (\frac{1}{2}(x + {x}^{T})) + \beta(\frac{1}{2}(y  + {y}^{T}))
%		\end{equation}
%		
%		\begin{equation}\label{q6aPrfBackSubn}
%			%L(\alpha x + \beta y) 
%			= \alpha L(x) + \beta L(y)
%		\end{equation}
%		
%		Thus, ${Eq}^{n}$ ~\ref{q6aPrfBackSubn} proves Statement~\ref{q6aLinOpStmt}. $\therefore$ the given operator $L$ is actually a \textit{Linear Operator}. \\ \textbf{Q.E.D.}
%		
%		\item[Question: 6.(b)] \setcounter{equation}{0} 
%		\item[Answer:] Given a linear transformation $L : \mathcal{X} \rightarrow \mathcal{X}$ and a basis $E$ given by 
%		
%		${E}^{11} = \begin{bmatrix}1 & 0 \\ 0 & 0 \end{bmatrix}$, ${E}^{12} = \begin{bmatrix}0 & 1 \\ 0 & 0 \end{bmatrix}$, ${E}^{21} = \begin{bmatrix}0 & 0 \\ 1 & 0 \end{bmatrix}$, ${E}^{22} = \begin{bmatrix}0 & 0 \\ 0 & 1 \end{bmatrix}$ used on both copies of $\mathcal{X}$.
%%		Consider $x \in \mathcal{X}$ s.t. $x = {\alpha}_{1}{E}^{11} + {\alpha}_{2}{E}^{12} + {\alpha}_{3}{E}^{21} + {\alpha}_{4}{E}^{22}$ $\Rightarrow {[x]}_{E} = \begin{bmatrix} {\alpha}_{1} \\ {\alpha}_{2} \\ {\alpha}_{3} \\ {\alpha}_{4}\end{bmatrix}$
%%		
%		\\ \\ By the theorem, we know that ${A}_{i} = {[L({u}^{i})]}_{v}$. But in our case, $u = v = E$. So, we can rewrite it as ${A}_{i} = {[L({E}^{ij})]}_{E}$ where $\forall i,j\in \mathbb{N} \; | \; 1 \leq i,j \leq 2$
%		
%		But as it is given that $L : \mathcal{X} \rightarrow \mathcal{X} \Rightarrow L$ is the Identity Operation $Id$. Thus, ${[L({E}^{ij})]}_{E} = {[{E}^{ij}]}_{E}$.
%		
%		So, now we can form the columns of $A = \begin{bmatrix} {A}_{1} & {A}_{2} & {A}_{3} & {A}_{4}\end{bmatrix}$ to be as, 
%		\begin{equation}\label{q6bMatRepnForm}
%			A = \begin{bmatrix} {[{E}^{11}]}_{E} & {[{E}^{12}]}_{E} & {[{E}^{21}]}_{E} & {[{E}^{22}]}_{E}\end{bmatrix}
%		\end{equation}
%		
%		\begin{equation}\label{q6bfinalAns}
%			\therefore A = 
%				\begin{bmatrix} 
%					1 & 0 & 0 & 0 \\
%					0 & 1 & 0 & 0 \\
%					0 & 0 & 1 & 0 \\
%					0 & 0 & 0 & 1
%				\end{bmatrix}
%		\end{equation}
		
%
		\item[Question: 7.(a)] \setcounter{equation}{0} 
		\item[Answer:] 
		
		\item[Question: 7.(b)] \setcounter{equation}{0} 
		\item[Answer:] 
		
	\end{qalist}
\end{document}