\documentclass[a4paper]{article}
\usepackage{enumitem, amsmath, gensymb, graphicx, caption, amssymb, geometry, fancyhdr, arydshln, adjustbox}

\geometry{left=1in, right=1in, top=1in, bottom=1in}
\pagestyle{fancy}

\newcommand{\myName}{\textbf{Shantanu Ghodgaonkar}\\\textit{Univ ID}: N11344563\\\textit{Net ID}: sng8399\\\textit{Ph.No.}: +1 (929) 922-0614}
\newlist{qalist}{description}{1}
\setlist[qalist]{style=unboxed,leftmargin=0.5cm,labelwidth=2.5cm}


\title{Homework 2 Answers : ROB-GY 6003}
\author{\myName}
\date{\today}

\fancyhead{} % Clear existing header settings 
\fancyhead[L]{\today}
\fancyhead[R]{N11344563}


\begin{document}
	
%	\begin{titlepage}
%	    \centering
%	    \vspace{2cm}
%	    \Huge\textbf{Mathematics for Robotics \\ ROB-GY 6103 \\ Homework 3 Answers}
%	    \vspace{1cm}
%	    \\ \Large \today
%	    \vfill
%	    \Large \myName
%	\end{titlepage}
	
	\begin{qalist}			
		\item[Question: 1.(a)] \setcounter{equation}{0} %Nagy, Page 136, Prob. 4.4.3
		\item[Answer:] Given, $V = {\mathbb{P}}_{2}$ with the ordered basis $\mathcal{S} = \left( {p}_{0} = 1,\;{p}_{1} = x,\;{p}_{2} = {x}^{2}\right)$ 
		
		And the given polynomial is $r(x) = 2 + 3x - {x}^{2}$ is, 
		
		
		$\therefore$ the components of $r(x)$ in basis $\mathcal{S}$ is 
		
		\begin{equation}
			{\text{r}}_{\mathcal{S}} = \begin{bmatrix} 2 \\ 3 \\ -1 \end{bmatrix}
		\end{equation}
		
		\item[Question: 1.(b)] \setcounter{equation}{0} %Nagy, Page 136, Prob. 4.4.3
		\item[Answer:] 
%
%		\item[Question: 2.] \setcounter{equation}{0} 
%		\item[Answer:] 
%
%		\item[Question: 3.] \setcounter{equation}{0} 
%		\item[Answer:] 
%
%		\item[Question: 4.] \setcounter{equation}{0} 
%		\item[Answer:] 
%
%		\item[Question: 5.] \setcounter{equation}{0} 
%		\item[Answer:] 
%
%		\item[Question: 6.] \setcounter{equation}{0} 
%		\item[Answer:] 
%
%		\item[Question: 7.] \setcounter{equation}{0} 
%		\item[Answer:] 
		
	\end{qalist}
\end{document}